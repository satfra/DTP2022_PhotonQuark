\begin{frame}
	\frametitle{Conclusion - What did we do?}
			\begin{itemize}
				\item We calculated the dressing functions of the quark-photon-vertex with the inhomogenious Bethe Salpeter equation
					\vspace{2mm}
				\item We used the coresponding Ward-Takahashi identity to check gaugeinvariance
					\vspace{2mm}
				\item We implemented Pauli-Villars and hard-cutoff regulation of our loopintegral 
					\vspace{2mm}
				\item Stephan provided us with Quark-DSE results 
			\end{itemize}
\end{frame}

\begin{frame}
	\frametitle{Conclusion - How we did it} 
			\begin{itemize}
				\item Straight forward!
					\vspace{2mm}
				\item we splitted different tasks at the very beginning
					\vspace{2mm}
				\item started implementation without asking questions
					\vspace{2mm}
				\item went for a time consuming and exhausting bughunt
			\end{itemize}
\end{frame}


\begin{frame}
    \frametitle{Conclusion - What did wie learn?}
    \begin{itemize}
        \item Prior numerical considerations are important and would have saved a lot of time!
        	\begin{itemize}
        		\item eg. using optimal grids to resolve structure
        	\end{itemize}
        		\vspace{3mm}
        \item Working together gets easier by alot if everyone is on the same page what to do!
        	\begin{itemize}
        		\item eg. reading the project description together to clarify difficulties    
        	\end{itemize}
    \end{itemize}
\end{frame}