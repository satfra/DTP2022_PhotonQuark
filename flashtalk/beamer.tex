\documentclass[pt12]{beamer}
%\documentclass[pt12,externalviewer]{beamer}

\usepackage{tikz}
\usetikzlibrary{calc}
\usepackage{verbatim}
\usepackage{amssymb}
\usepackage{graphicx}
%\usepackage{rotating}
%\usepackage{amsmath}
%\usepackage{tikz}
%\usepackage{beamergraphics}
\usepackage[]{babel}
%\usepackage[latin1]{inputenc}
\usepackage[T1]{fontenc}
\newcommand{\av}[1]{\langle #1 \rangle}
\newcommand{\mel}[3]{\langle#1|#2|#3\rangle}
\newcommand{\ket}[1]{|#1\rangle}
\newcommand{\scal}[2]{\langle #1|#2\rangle}
\newcommand{\ord}[1]{^{(#1)}}
\newcommand{\bra}[1]{\langle #1|}
\newcommand{\limit}[2]{\lim_{#1 \to #2}}

%\usepackage[absolute,overlay]{textpos}
%\usepackage{pdfcolparallel}

%\usepackage{tcolorbox}
%\tcbuselibrary{fitting}

\definecolor{CTgreen}{HTML}{28794e} % Catania green (primary)
\definecolor{UBCblue}{rgb}{0.02706, 0.04725, 0.526667} % UBC Blue (primary)
%\definecolor{UBCblue}{rgb}{0.01706, 0.03725, 0.426667} % UBC Blue (primary)
\mode<presentation>
{
  \usetheme{Warsaw} 
%  \usetheme{Madrid}
%  \usetheme{Montpellier}
%  \usetheme{Marburg} 
  \usecolortheme[named=UBCblue]{structure}
  \setbeamercolor{alerted text}{fg=blue}
  \setbeamercovered{transparent}
  \setbeamertemplate{section in toc}[ball unnumbered]
}

%\setbeamertemplate{footline}{\hfill\insertframenumber/\inserttotalframenumber} 

\expandafter\def\expandafter\insertshorttitle\expandafter{%
  \insertshorttitle\hfill%
  \insertframenumber\,/\,\inserttotalframenumber}

\newcommand{\backupbegin}{
   \newcounter{framenumberappendix}
   \setcounter{framenumberappendix}{\value{framenumber}}
}
\newcommand{\backupend}{
   \addtocounter{framenumberappendix}{-\value{framenumber}}
   \addtocounter{framenumber}{\value{framenumberappendix}} 
}

\newcommand{\refer}[1]{%
   \begin{flushright}
      {\alert{\tiny #1}}
   \end{flushright}}
  
\newcommand{\lrefer}[1]{%
   \begin{flushleft}
      {\alert{\tiny #1}}
   \end{flushleft}}
  
\newcommand{\param}[1]{%
   \begin{flushright}
      {\small #1}
   \end{flushright}
   \vspace{-1.5\baselineskip}
}

\newcommand{\sech}{\mathop{\rm sech}\nolimits}
\newcommand{\sgn}{\mathop{\rm sgn}\nolimits}
\newcommand{\etal}{{\em et al.}}


\title[Pulsars: properties and glitches]{Quark-Photon Vertex}


\author{S. Hagel, L. Kiefer, F. Murgana, F.R. Sattler, J. Wessely}




\date{18-05-2022}

\begin{document}

\begin{frame}[plain]
\titlepage
\end{frame}



\begin{frame}[label=outline]
\frametitle{Outline}
\tableofcontents[pausesections]
\end{frame}

\section{Motivation}
\begin{frame}
    \frametitle{Motivations}
    \begin{itemize}
        \item Electromagnetic form factor;
        \item Anomalous magnetic moment;
    \end{itemize}
    \vspace{2mm}

    The coupling of a photon to an on-shell spin-$1/2$ fermion with mass $m$ is described by the electromagnetic
    current matrix element
    \begin{equation}
        \label{eq:1}
        J^\mu(k, Q)=i\bar{u}(k_+)\left[F_1(Q^2)\gamma^\mu-\frac{F_2(Q^2)}{2m}\sigma^{\mu\nu}Q^\nu\right]u(k_-)
    \end{equation}
\end{frame}
\endinput


\section{Quark-Photon Vertex}

\begin{frame}\frametitle{Quark-Photon Vertex}

The quark-photon vertex must satisfy electromagnetic gauge invariance in the form of the
Ward-Takahashi identity (WTI):

\begin{equation}
	Q^\mu\Gamma^\mu(k, Q)=S(k_+)^{-1}-S(k_-)^{-1}
\end{equation}

\begin{minipage}[r]{0.65\textwidth}
	$\rightarrow$ Split vertex in transverse and longitudinal parts $T_i^{\mu} \; \text{and} \; G_j^{\mu} $ with respect to incoming photon momentum:
\end{minipage}
\begin{minipage}[r]{0.30\textwidth}
	\hspace{2mm}
	\includegraphics[height=2.2cm, width=3.2cm]{Vertex.png}
\end{minipage}

\begin{equation}
	\Gamma^\mu(k,Q)=\sum_{j=1}^4 g_j(k^2, \omega, Q^2)iG^\mu_j(k, Q)+\sum_{j=1}^8 f_j(k^2, \omega, Q^2)iT^\mu_j(k, Q)
\end{equation}

\end{frame}

\begin{frame}\frametitle{Quark-Photon Vertex}
With this decomposition, the vertex has a charge-conjugation symmetry

\begin{equation}
	\bar{\Gamma}^\mu(k,Q):=-C\Gamma^\mu(-k,Q)^TC^T=\Gamma^\mu(k, -Q) \qquad C=\gamma^4\gamma^2
\end{equation}
\vspace{4mm}
It is easy to show that each $G_j^\mu$, $T_j^\mu$
satisfies the same relation $\Rightarrow$\\ \vspace{3mm} $g_j(k^2, \omega, Q^2)=g_j(k^2, \omega^2, Q^2)$, $\qquad$ $f_j(k^2, \omega, Q^2)=f_j(k^2, \omega^2, Q^2)$\\

\vspace{9mm}

The tensor basis  is free of kinematic constraints $\Rightarrow$
\vspace{2mm}
$g_j(k^2, \omega^2,Q^2)$ and $f_j(k^2, \omega^2,Q^2)$ become constant for $Q^\mu\to0$ or $k_\mu\to  0$.

\end{frame}


\endinput


\section{Bethe-Salpeter equation (BSE)}
\begin{frame}\frametitle{Bethe-Salpeter equation}
To calculate the quark-photon vertex dynamically, we can solve its Bethe-Salpeter equation:
\begin{figure}[h]
	\centering
	\includegraphics[height=2.7cm, width=10.2cm]{BSE.png}
\end{figure}
\textit{Rainbow-ladder truncation}: approximate the full kernel by
a gluon exchange with an effective interaction given by the Maris-Tandy model 


\end{frame}

\begin{frame}\frametitle{Bethe-Salpeter equation}
	
\begin{equation}
	g(k^2)=Z^2_2\frac{16\pi}{3}\frac{\alpha(k^2)}{k^2}
\end{equation}	

\begin{equation}
\alpha(k^2)=\pi \eta^7 x^2 e^{-\eta^2 x} + \frac{2\pi\gamma_m\left(1-e^{k^2/\Lambda_t^2}\right)}{\ln\left[e^2 - 1 + \left(1 + \frac{k^2}{\Lambda_{QCD}^2}\right)^2\right]}, \qquad x=\frac{k^2}{\Lambda^2}
\end{equation}

The resulting BSE reads explicitly:
\begin{equation}
	\Gamma^\mu(k, Q)=Z_2i\gamma^\mu+\int_{k'}\!\!g(l^2)T^{\alpha\,\beta}_l\gamma^\alpha S(k'_+)\Gamma^\mu(k', Q)S(k'_-)\gamma^\beta
\end{equation}
\end{frame}

\endinput


\begin{frame}\frametitle{Basistransformation}
\begin{minipage}[r]{1.\textwidth}
	\hspace{2mm}
	\includegraphics[scale=0.8]{gs.pdf}
	\vspace{4mm}
	\includegraphics[scale=0.7]{fs.png}
\end{minipage}

\small The transformation to the longitudinal/transverse Dressingfunctions

\vspace{3mm}
%



\end{frame}


\endinput

\include{slide5}

\begin{frame}\frametitle{How did we do it?}
  \setbeamercovered{invisible}
  \begin{columns}
    \begin{column}{0.4\linewidth}
      \begin{block}{Tasks:}
        \begin{itemize}
          \item Implement $K_{ij}$, $G_{ij}$
          \item Convenience functions
          \item 2d integration
          \item 2d interpolation
          \item Maris-Tandy
          \item Quark $M,\, A$
        \end{itemize}
      \end{block}
    \end{column}
    \pause
    \begin{column}{0.2\linewidth}
      \centering
      $\rightarrow$
    \end{column}
    \begin{column}{0.4\linewidth}
      \begin{block}{New tasks:}
        \begin{itemize}
          \item Iteration
          \item Calculate $K'_{ij}$
          \item File output
          \item Optimize
        \end{itemize}
      \end{block}
    \end{column}
  \end{columns}
  \pause
  \begin{tikzpicture}[remember picture,overlay]
    \node[anchor=south west,inner sep=0pt,text width=10cm] at ($(current page.south west)+(5.2cm,5.99cm)$) {
      \fontsize{12}{14}\selectfont\textcolor{black}{Around two days...}
    };
  \end{tikzpicture}
  \pause
  \begin{tikzpicture}[remember picture,overlay]
    \node[anchor=south west,inner sep=0pt] at ($(current page.south west)+(8.2cm,6.cm)$) {
      \includegraphics[width=0.18\linewidth]{Important.jpeg}
    };
  \end{tikzpicture}
  \begin{tikzpicture}[remember picture,overlay]
    \node[anchor=south west,inner sep=0pt,text width=10cm] at ($(current page.south west)+(7.2cm,1.0cm)$) {
      \fontsize{30}{35}\selectfont\textcolor{red}{\textbf{Bugs.}}
    };
  \end{tikzpicture}
\end{frame}

\begin{frame}\frametitle{Bugs}
  \setbeamercovered{invisible}
  {\Large Collective Bughunt:}


  {\small
    \begin{tikzpicture}[remember picture,overlay]
      \node[anchor=south west,inner sep=0pt] at ($(current page.south west)+(7.2cm,5.0cm)$) {
        \includegraphics[width=0.3\linewidth]{frog.png}
      };
    \end{tikzpicture}

    \pause
    \vspace{3mm}
    We found bugs in Normalization $\rightarrow$ factor $(2\pi)^4 \approx 10^3$

    \pause
    \vspace{3mm}
    Forgot Jacobi determinant

    \pause
    \vspace{3mm}
    Mismatched some indices
  }
\end{frame}

\begin{frame}\frametitle{Stupid numerics}
  \setbeamercovered{invisible}
  \begin{columns}
    \centering
    \begin{column}{0.4\linewidth}
      \begin{block}{}
        \vspace{-0.cm}
        But still no convergence, everything goes to $\infty$...
      \end{block}
    \end{column}
    \pause
    \begin{column}{0.2\linewidth}
      \centering
      \tiny
      Some parameter testing...\\
      $\rightarrow$
    \end{column}
    \pause
    \begin{column}{0.4\linewidth}
      \begin{block}{Numerics issue}
        \vspace{-0.cm}
          Oversampled UV tail
      \end{block}
    \end{column}
  \end{columns}
  \pause
  \vspace{1cm}
  \centering
  $\Rightarrow$ Logarithmic grid for $k^2$
\end{frame}

\begin{frame}\frametitle{Program overview}
  \tiny
      \begin{figure}
        \begin{forest}
          for tree={
            align=center,
            font=\sffamily,
            edge+={thin, -{[]}},
            l sep'+=10pt,
            fork sep'=10pt,
          },
          forked edges,
          if level=0{
            inner xsep=0pt,
            tikz={\draw [thick] (.children first) -- (.children last);}
          }{},
          [Quark\_Photon\_Vertex
            [Simulation.cpp
              [iteration.hh
                [
                [quark\_dse.hh]
                [Kernels\_G.hh]
                [Kernels\_K.hh]
                [maris\_tandy.hh]
                ]
                [
                [
                [quark\_model\_functions.hh]
                [basistransform.hh]
                [momentumtransform.hh]
                ]
                ]
              ]
              [parameters.hh]
              [types.hh]
              [fileIO.hh]
            ]
          ]
        \end{forest}
      \end{figure}
      \begin{figure}
        \begin{forest}
          for tree={
            align=center,
            font=\sffamily,
            edge+={thin, -{[]}},
            l sep'+=10pt,
            fork sep'=10pt,
          },
          forked edges,
          if level=0{
            inner xsep=0pt,
            tikz={\draw [thick] (.children first) -- (.children last);}
          }{},
          [core
            [PolynomialBase.hh
              [LegendrePolynomials.hh]
            ]
            [LinearInterpolate.hh]
            [QuadratureIntegral.hh]
            [Utils.hh]
          ]
        \end{forest}
      \end{figure}
\end{frame}
\endinput
\section{Pulsar Glitches and Superfluidity}

% \begin{frame}\frametitle{Solving the BSE}
 	


We rewrite the decomposition as 
\small\begin{equation}
\!\!\!\!\!\!\!	\Gamma^\mu(k, Q)=\sum_{j=1}^{12} F_j(k^2, \omega, Q^2)it^\mu_j(k,Q), \quad F_j\in\{g_j,f_j\}, \; t^\mu\in\{G^\mu_j, T^\mu_j\}
\end{equation}

$$\!\!\!\!\!\!\!\!\!H_{ij}(k^2, \omega, Q^2)=1/4\mbox{Tr}\left\{\bar{t}^\mu_i(k,Q)t^\mu_j(k,Q)\right\}\neq\delta_{ij}\Longrightarrow t^\mu_j  \mbox{ not ortonormal }$$
We want to construc an orthonormal basis $H_{ij}=\delta_{ij}$ 
Fisrt we choose a frame for the four vectors:
\begin{figure}[!htb]
	
	\minipage{0.32\textwidth}
	\hspace{-7mm}
	\includegraphics[width=4cm, height=2cm]{kps.png}

	\endminipage\hfill
	\minipage{0.32\textwidth}
	\includegraphics[width=\linewidth]{ks.png}

	\endminipage\hfill
	\minipage{0.32\textwidth}%
	\includegraphics[width=\linewidth,  height=2.3cm]{Qs.png}
	
	\endminipage
\end{figure}



\end{frame}

\begin{frame}
	In this way we can express the quark-photon vertex in the following basis:
	\begin{equation}
		\Gamma^\mu(k, Q)=\sum_{j=1}^{12}a_(k^2, \omega, Q^2)i\tau_j^\mu(k,Q)
	\end{equation}
with
\begin{figure}
		\includegraphics[width=\linewidth,  height=2.3cm]{taus.png}
\end{figure}
\end{frame}

\endinput


\section{Starquakes}
%\include{slide5}

%\include{conclusions}

%\appendix
%\backupbegin

%\section*{Parte ausiliaria}
%\include{slideaux1}




\end{document}

